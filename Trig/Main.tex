\documentclass[a4paper, 12pt]{article}


\usepackage{xcolor, sectsty}
\definecolor{dark-red}{RGB}{139, 0, 0}
\definecolor{firebrick}{RGB}{178, 34, 34}
\definecolor{crimson}{RGB}{220, 20, 60}
\definecolor{background}{RGB}{236, 231, 238}
\definecolor{text}{RGB}{2, 4, 18}
\definecolor{darkblue}{RGB}{22, 22, 138}

\sectionfont{\huge\color{crimson}}
\subsectionfont{\LARGE\color{firebrick}}
\pagecolor{background}

\renewcommand{\thesection}{\large\Roman{section})}
\renewcommand{\thesubsection}{\Large\arabic{section}°)}
\renewcommand{\Large}{\color{text}}
\renewcommand{\textbf}{\color{crimson}}
\renewcommand{\theenumi}{\color{crimson}\Roman{enumi}}

\title{\color{dark-red}\huge\textbf{Formules de trigonométrie}}
\date{15-10-2022}
\author{Quentin RIGGI}
\pagenumbering{Roman}

\usepackage{geometry, amssymb, amsmath, scalerel, hyperref, multirow, mathtools, marginnote}
\geometry{a4paper, total={171mm,257mm}, left=20mm, top=20mm}

\begin{document}
	\maketitle
	\section{Addition}
	\vspace{-1cm}
	\begin{center}
		\item \subsection*{Propriétés}
	
		\vspace{-0.4cm}

		\normalsize\begin{gather*}
			\begin{split}
				\forall (a, b)\in\mathbb{R}^2:\;&cos\left(a+b\right) = \cos{a}\times\cos{b} - \sin{a}\times\sin{b}\\
				&cos\left(a-b\right) = \cos{a}\times\cos{b} + \sin{a}\times\sin{b}\\
				&sin\left(a+b\right) = \sin{a}\times\cos{b} + \sin{b}\times\cos{a}\\
				&sin\left(a-b\right) = \sin{a}\times\cos{b} - \sin{b}\times\cos{a}\\
				\\
				a+b\neq\;\frac{\pi}{2}+k\pi/k\in\mathbb{Z}\\
				tan\left(a+b\right) =&\, \frac{sin\left(a+b\right)}{cos\left(a+b\right)} = \frac{\sin{a}\times\cos{b}+\sin{b}\times\cos{a}}{\cos{a}\times\cos{b}-\sin{a}\times\sin{b}}\\
				=&\, \frac{\cos{a}\times\cos{b}\times\left(\frac{\sin{a}\times\cos{b}}{\cos{a}\times\cos{b}}+\frac{\sin{b}\times\cos{a}}{\cos{a}\times\cos{b}}\right)}{\cos{a}\times\cos{b}\times\left(\frac{\cos{a}\times\cos{b}}{\cos{a}\times\cos{b}}-\frac{\sin{a}\times\sin{b}}{\cos{a}\times\cos{b}}\right)}\\
				=&\,\frac{\tan{a}+\tan{b}}{1-\tan{a}\times\tan{b}}\\
			\end{split}
		\end{gather*}

		\vspace{0.2cm}

		\item \subsection*{Duplication}
		\vspace{-0.4cm}
		\begin{gather*}
			\begin{split}
				x &= a = b\\
				\cos{2x} &= \cos^2{x}-\sin^2{x}\\
				&= 1 - 2\times\sin^2{x}\\
				&= 2\times\cos^2{x} - 1\\
				\\
				\sin{2x} &= 2\times\sin{x}\times\cos{x}\\
				\\
				\tan{2x} &= \frac{2\times\tan{x}}{1-\tan^2{x}}
			\end{split}
		\end{gather*}
	\end{center}
	
	\vspace{2cm}
	\section{Produit vers la somme}
	\vspace{-0.4cm}
	\begin{gather*}
		\begin{split}
			\cos{a}\times\cos{b} &= \frac{1}{2}\times\left[cos\left(a+b\right) + cos\left(a-b\right)\right]\\
			\sin{a}\times\sin{b} &= \frac{1}{2}\times\left[cos\left(a+b\right) - cos\left(a-b\right)\right]\\
			\sin{a}\times\cos{b} &= \frac{1}{2}\times\left[sin\left(a+b\right) + sin\left(a-b\right)\right]\\
			\\
			\cos^2{x} &= \frac{\cos{2x}+1}{2}\\
			\sin^2{x} &= \frac{1-\cos{2x}}{2}
		\end{split}
	\end{gather*}

	\vspace{0.5cm}
	\section{Somme vers le produit}
	\begin{gather*}
		\begin{split}
			On\;pose:&
			\begin{cases}
				p = a + b\\
				q = a - b	
			\end{cases}
			\Leftrightarrow
			\begin{cases}
				a = \frac{p+q}{2}\\
				b = \frac{p-q}{2}
			\end{cases}
			\\\\
			\cos{p}+\cos{q} &= 2\times\cos{\frac{p+q}{2}}\times\cos{\frac{p-q}{2}}\\
			\cos{p}-\cos{q} &= 2\times\sin{\frac{p+q}{2}}\times\sin{\frac{p-q}{2}}\\
			\sin{p}+\sin{q} &= 2\times\sin{\frac{p+q}{2}}\times\cos{\frac{p-q}{2}}
		\end{split}
	\end{gather*}

	\newpage
	\section{Résolution d'équation trigonométrique}
	\begin{center}
		\begin{split}
			
		\end{split}
	\end{center}
\end{document}
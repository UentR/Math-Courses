\documentclass[a4paper, 10pt]{article}


\usepackage{xcolor, sectsty}
\definecolor{dark-red}{RGB}{139, 0, 0}
\definecolor{firebrick}{RGB}{178, 34, 34}
\definecolor{crimson}{RGB}{220, 20, 60}
\definecolor{background}{RGB}{236, 231, 238}
\definecolor{text}{RGB}{2, 4, 18}
\definecolor{darkblue}{RGB}{22, 22, 138}

\sectionfont{\huge\color{crimson}}
\subsectionfont{\LARGE\color{firebrick}}
\pagecolor{background}

\renewcommand{\thesection}{\huge\Roman{section})}
\renewcommand{\thesubsection}{\LARGE\arabic{section}°)}
\renewcommand{\large}{\color{text}}
\renewcommand{\textbf}{\color{crimson}}
\renewcommand{\theenumi}{\color{crimson}\Roman{enumi}}

\title{\color{dark-red}\Huge LIMITES ET CONTINUITÉ DES FONCTIONS}
\date{03-09-2022}
\author{Quentin RIGGI}
\pagenumbering{Roman}

\usepackage{geometry, amssymb, amsmath, scalerel, hyperref, mathtools, marginnote}
\geometry{a4paper, total={171mm,257mm}, left=30mm, right=30mm, top=30mm, bottom=20mm}

\usepackage{tikz}
\usetikzlibrary{patterns,hobby}
\usepackage{pgfplots}
\pgfplotsset{compat=1.6}

\begin{document}
	\maketitle
	\large L'objectif est d'étudié le comportement des valeurs f(x) prisent par une fonction f aux bornes ouvertes de son domaine de définition.\\
	\large On introduit une nouvelle notion, celle de la continuité d'une fonction, plus forte qu'être définie mais plus faible que dérivable.
	
	
	\section{Limite en l'infini et droite asymptote}	
	\large Par la suite on considère une fonction f, dont le domaine de définition Df contraint une contient une borne $+\infty$ 

	\begin{center}
		\subsection*{Limite infinie en l'infini}	
	\end{center}
	\textbf{Definition:}
	\large On dit qu'une fonction f tend vers +$\infty$ lorsque x tend vers +$\infty$, si tout intervalle ouvert de la forme
	$\left]A; +\infty \right[$ où $A\in\mathbb{R}$ contient toutes les valeurs f(x) prisent par la fonction f dès que x est choisi suffisamment grand.\\
	\large On note:
	\begin{displaymath}
		\scaleto{\lim_{x\to+\infty} f(x) = +\infty}{25pt}
	\end{displaymath}
	\\
	\large Avec les quantificateurs:
	\begin{gather*}
		\scaleto{\lim_{x\to+\infty} f(x) = +\infty}{25pt} 
		\\
		\scaleto{\Leftrightarrow\forall A \in\mathbb{R}_{+}^{*},\exists\beta\in\mathbb{R}_{+}^{*},\forall x \in D_{f} : \left(x\geq\beta\Rightarrow f\left(x\right)\geq A \right)}{18pt}
	\end{gather*}
	
	%Curve
	\begin{center}
		\begin{tikzpicture}
			\begin{axis}[
				xmin=0,xmax=5,
				xlabel={z},
				ymin=0,ymax=5,
				xtick={3},
				xticklabels={$\beta$},
				ytick={3},
				yticklabels={$A$},
				xlabel={$x$},  
				ylabel={$y$},
				axis lines=middle]
				\addplot[black,thick,domain=0:5,no marks]{3};
				\addplot +[black,thick,dashed,domain=0:5,no marks] coordinates {(3,0) (3,3)};
			\end{axis}
			\draw [thick] (0, 0) .. controls(0.9,5) and (3.4,1.6) .. (6.5, 5);
		\end{tikzpicture}
	\end{center}
	
	\newpage
	\textbf{Définition:}
	\large On dit qu'une fonction f tend vers -$\infty$ lorsque x tend vers +$\infty$
	si tout intervalle ouvert de la forme $\left]-\infty; A\right[, A\in\mathbb{R}$, contient toutes les valeurs f(x)
	prisent par la fonction f dès que x est choisi assez grand.\\
	On note:
	\begin{displaymath}
		\scaleto{\lim_{x\to+\infty} f(x) = -\infty}{25pt}
	\end{displaymath}
	\\
	\large Avec les quantificateurs:
	\begin{gather*}
		\scaleto{\lim_{x\to+\infty} f(x) = -\infty}{25pt} 
		\\
		\scaleto{\Leftrightarrow\forall A \in\mathbb{R}_{-}^{*},\exists\beta\in\mathbb{R}_{+}^{*},\forall x \in D_{f} : \left(x\geq\beta\Rightarrow f\left(x\right)\leq A \right)}{18pt}
	\end{gather*}

	
	\begin{center}
		\vspace{30pt}
		\small On a aussi les deux autres limites:
		\begin{gather*}
			\scaleto{\lim_{x\to-\infty} f(x) = +\infty\Leftrightarrow\forall A \in\mathbb{R}_{+}^{*},\exists\beta\in\mathbb{R}_{+}^{*},\forall x \in D_{f} : \left(x\leq\beta\Rightarrow f\left(x\right)\geq A \right)}{18pt}
			\\
			\scaleto{\lim_{x\to-\infty} f(x) = -\infty\Leftrightarrow\forall A \in\mathbb{R}_{-}^{*},\exists\beta\in\mathbb{R}_{+}^{*},\forall x \in D_{f} : \left(x\leq\beta\Rightarrow f\left(x\right)\leq A \right)}{18pt}
		\end{gather*}
	\end{center}

	\vspace{30pt}
	\textbf{Propriétés (Admises):}
	\begin{gather}
		\color{text} \scaleto{\forall n\in\mathbb{N}^{*}: \lim_{x\to+\infty}x^{n}=+\infty}{20pt} \label{xn} 
		\\ \color{text} \scaleto{\forall n\in2\mathbb{N}, n\neq0: \lim_{x\to-\infty}x^{n}=+\infty}{20pt}
		\\ \color{text} \scaleto{\forall n\in2\mathbb{N}+1: \lim_{x\to-\infty}x^{n}=-{\infty}}{20pt}
		\\ \color{text} \scaleto{\lim_{x\to+\infty}\exp{x}=+\infty}{20pt}
		\\ \color{text} \scaleto{\nexists\lim_{x\to\pm\infty}\cos{x}}{20pt} 
		\\ \color{text} \scaleto{\nexists\lim_{x\to\pm\infty}\sin{x}}{20pt}
	\end{gather}

	\bigskip
	\textbf{\color{darkblue} Exemple:}\\
	\large La fonction x $\mapsto x^{2}$ est définie sur $\mathbb{R}$\\
	Soit A$\in\mathbb{R}_{+}^{*}$, on cherche un réel $\beta$ tel que:
	\begin{gather*}
		\begin{split}
			x\geq\beta&\Rightarrow x^{2}\geq A
			\shortintertext{Soit:} 
			x^{2}\geq A &\Leftrightarrow\lvert x\rvert\sqrt{A}\\
			&\Leftrightarrow x\in\left]-\infty; -\sqrt{A}\right]\cup\left[\sqrt{A}; +\infty\right[\\
			\shortintertext{On pose $\beta$ = $\sqrt{A}$}
			\forall A\in\mathbb{R}_{+}^{*}, \exists\beta&=\sqrt{A}\in\mathbb{R}_{+}^{*}, \forall x\in\mathbb{R}: x\geq\beta\Rightarrow f(x)\geq A
		\end{split}
	\end{gather*}


	\newpage
	\begin{center}
		\subsection*{Limites finie en l'infini et asymptote horizontale}
	\end{center}
	\textbf{Définition:}
	\large Soit l un nombre réel.\\
	On dit que la fonction f tend vers l, lorsque x tend vers +$\infty$ fi tout intervalle ouvert contenant
	l contient toutes les valeurs f(x) prisent par la fonction f dès que x est choisi suffisamment grand.\\
	On note:
	\begin{displaymath}
		\scaleto{\lim_{x\to+\infty}f(x)=l}{22pt}\\
		\scaleto{\Leftrightarrow\exists l\in\mathbb{R},\forall\varepsilon\in\mathbb{R}_{+}^{*},\exists x\in D_{f}: x\geq\beta\Rightarrow\lvert f(x)-l\lvert\leq\varepsilon}{16pt}
	\end{displaymath}

	
	\textbf{Remarque :}
	\begin{enumerate}
		\item \large Graphiquement, cette définition signifie que dans un repère orthogonal la courbe $C_{f}$ représentative de la fonction f admet une asymptote horizontale d'équation y = l au voisinage de +$\infty$
		\item \large On défini de même l'asymptote horizontale au voisinage de -$\infty$
		\item \large Si la limite de f(x) quand x tend vers $+\infty$ vaut l on dit simplement que $C_{f}$ admet une asymptote horizontale d'équation y = l au voisinage de l'infini $\left(-\infty, +\infty\right)$
	\end{enumerate}
	
	\bigskip
	\textbf{\color{darkblue}Exemple:}
	\large Soit f la fonction définie par f(x) = 2 - $\frac{5}{x^2}$\\
	Montrez qu'il existe un réel $x_0$ tel que x $>$ $x_0 \Rightarrow$ 1,95 $<$ f(x) $<$ 2,05\\
	\newline
	$D_f$ = $\mathbb{R}^*$
	\begin{gather*}
		\begin{split}
			\shortintertext{$\forall$ x$\in$$D_f$}
			1,95 < f(x) < 2,05 &\Leftrightarrow -0,05 < f(x) - 2 < 0,05\\
			&\Leftrightarrow-0,05 < \frac{-5}{x^2} < 0,05^1
			\marginnote{\normalsize $^1$On pose:\\$\varepsilon$ = 5$\times10^{-2}$}
			\\
			&\Leftrightarrow-0,01 < \frac{1}{x^2} < 0,01\\
			&\Leftrightarrow -10^{-2} < \frac{1}{x^2}< 10^{-2}\\
			&\Leftrightarrow
			\begin{cases}
				-10^{-2} < \frac{1}{x^2}, \forall x\in\mathbb{R}^*\\
				\frac{1}{x^2} < 10^{-2}
			\end{cases}
			\\
			\frac{1}{x^2} < 10^{-2}&\Leftrightarrow x^2 > 10^2\\
			&\Leftrightarrow\lvert x\rvert > 10\\
			&\Leftrightarrow x\in \left]-\infty; -10\right] \cup \left[10; +\infty\right[\\
			\shortintertext{En posant $x_0$ = 10}
			\forall\varepsilon\in\mathbb{R}_{+}^{*},\exists x_0=10\in\mathbb{R},\forall x\in\mathbb{R}^*: x>x_0 &\Leftrightarrow 1,95 < f(x) < 2,05\\
			&\Leftrightarrow\lvert f(x) - 2\rvert < \varepsilon
		\end{split}
	\end{gather*}

	\textbf{Graph:}
	\large Dans un repère (O, $\vec{i}$, $\vec{j}$), la courbe $C_f$ admet la droite d'équation y = 2 pour asymptote horizontale au voisinage de +$\infty$.
	\vspace{0.8cm}
	
	\textbf{Propriétés:}
	\begin{gather}
		\color{text} \scaleto{\forall n\in\mathbb{N}^*\lim_{x\to\pm\infty}\frac{1}{x^n}=0}{25pt}\\
		\color{text} \scaleto{\lim_{x\to-\infty}\exp{x}=0}{20pt}
	\end{gather}
	\newpage

	\begin{center}
		\subsection*{Asymptote oblique}
	\end{center}

	
\end{document}
\documentclass{article}

\usepackage{xcolor, sectsty}
\definecolor{darkred}{RGB}{139, 0, 0}
\definecolor{firebrick}{RGB}{178, 34, 34}
\definecolor{crimson}{RGB}{220, 20, 60}

\sectionfont{\huge\color{crimson}}
\subsectionfont{\LARGE\color{firebrick}}

\renewcommand\thesection{\huge\Roman{section})}
\renewcommand\thesubsection{\LARGE\arabic{section}°)}

\title{\color{darkred}\Huge LIMITES ET CONTINUITÉ DES FONCTIONS}
\date{03-09-2022}
\author{Quentin RIGGI}

\usepackage{geometry}
\geometry{a4paper, total={170mm,257mm}, left=20mm, top=20mm,}

\begin{document}
	\maketitle
	\large L'objectif est d'étudié le comportement des valeurs f(x) prisent par une fonction f aux bornes ouvertes de son domaine de définition.\\
	\large On introduit une nouvelle notion, celle de la continuité d'une fonction, plus forte qu'être définie mais plus faible que dérivable.
	
	\section{Limite en l'infini et droite asymptote}
	\paragraph{} \large Par la suite on considere une fonction f, dont le domaine de définition Df contraint une contient une borne $+ \infty$ 

	\subsection{Limite infinie en l'infini}
	\textbf{\color{crimson}Definition:} On dit qu'une fonction f tend vers +$\infty$ lorsque x tend vers +$\infty$, si tout intervalle ouvert de la forme
	$\left]A; +\infty \right[$ où $A\in$ \math{R}$
	
\end{document}